\section{Installing the \pate{} Verifier}
\label{sec:build-pate-verif}

We recommending using \pate{} via Docker.

\subsection{Loading a Docker Image File}

If available, load the Docker image file from a file using:
\begin{verbatim}
  docker load -i /path/to/pate.tar
\end{verbatim}

\subsection{Building the Docker Container}

To build the Docker image from the \pate{} source repo, use the following commands:

\begin{verbatim}
  git clone git@github.com:GaloisInc/pate.git
  cd pate
  git submodule update --init
  docker build . -t pate
\end{verbatim}

Subsequently, \pate{} may be used via Docker, such as:

\begin{verbatim}
  docker run --rm -it \
    -v "$(pwd)"/tests:/tests pate \
    --original /tests/aarch32/const-args.original.exe \
    --patched /tests/aarch32/const-args.patched.exe
\end{verbatim}

Please see later sections for detailed usage information.

\subsection{(Optional): Building from Source}

Alternatively, \pate{} may be compiled from source and run locally.
The \pate{} tool is written in Haskell and requires the GHC compiler\footnote{\url{https://www.haskell.org/ghc/}} and the cabal\footnote{\url{https://www.haskell.org/cabal/}} to build.

We have tested with GHC version 9.6.2 and cabal version 3.10.2.0.
Additionally, the verifier requires an SMT solver to be available in \texttt{PATH}.
We recommend \texttt{yices}.
The \texttt{z3} and \texttt{cvc4} solvers may also work but are not regularly tested with \pate{}.

Building from source can be accomplished as follows:
\begin{verbatim}
  git clone git@github.com:GaloisInc/pate.git
  cd pate
  git submodule update --init
  cp cabal.project.dist cabal.project
  cabal configure pkg:pate
  ./pate.sh --help
\end{verbatim}

Once built, invoke pate locally using the \texttt{pate.sh} script:

\begin{verbatim}
  ./pate.sh --help
\end{verbatim}

%%% Local Variables:
%%% mode: latex
%%% TeX-master: "user-manual"
%%% End:
